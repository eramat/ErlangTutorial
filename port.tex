\section{Multilangage}

\begin{frame}[fragile]
  \frametitle{Multilangage}
  \framesubtitle{Introduction}

  \begin{block}{Définition}
    \begin{itemize}
    \item Erlang permet de définir des mécanismes de haut niveau : multitâche,
      distribué, \ldots
    \item Erlang n'est pas fait pour faire du calcul scientifique, par exemple
    \item il est donc intéressant de pouvoir ``communiquer'' avec du code
      non-Erlang
    \item le mécanisme propose se nomme : \textit{port}
    \item l'objet externe est vu comme un processus Erlang avec lequel on
      peut communiquer
    \item l'objet reçoit des messages et peut en envoyer au code Erlang
    \item l'objet est un exécutable
    \end{itemize}
  \end{block}

\end{frame}

\begin{frame}[fragile]
  \frametitle{Multilangage}
  \framesubtitle{Port}

  \begin{exampleblock}{Communications avec un port}
    Seulement 3 types de message sont compris par un port :
    \begin{itemize}
    \item envoi d'une trame d'octets Data à l'objet
      \begin{lstlisting}[language=erlang]
Port ! {PidC, {command, Data}}
      \end{lstlisting}
    \item change le processus connecté au port
      \begin{lstlisting}[language=erlang]
Port ! {PidC, {connect, Pid1}}
      \end{lstlisting}
    \item fermeture de la connexion avec l'objet
      \begin{lstlisting}[language=erlang]
Port ! {PidC, close}
      \end{lstlisting}
    \end{itemize}
  \end{exampleblock}

\end{frame}

\begin{frame}[fragile]
  \frametitle{Multilangage}
  \framesubtitle{Port}

  \begin{block}{Création}
    La fonction principale de création d'un port est
    \textit{open\_port(PortName, PortSettings)} où :
    \begin{itemize}
    \item \textit{PortName} peut être \textit{{spawn, Command}}, Command est
      alors un programme exécutable lancé dans un processus
    \item \textit{PortSettings} décrit le mécanisme de communication :
      \begin{itemize}
      \item \textit{use\_stdio} : utilise les entrée-sorties standards
      \item \textit{binary} : les échanges sont en binaire (dans ce cas, il
        existe des fonctions BIF pour faire les conversions)
      \item \textit{{packet, N}} : chaque message est une liste d'entiers
        précédée par la taille de la liste codée sur N octets (avec N = 1, 2
        ou 4)
      \end{itemize}
    \end{itemize}
  \end{block}

\end{frame}

\begin{frame}[fragile]
  \frametitle{Multilangage}
  \framesubtitle{\textit{Linked-in Driver}}

  \begin{alertblock}{Définition}
    Il est possible de lier de manière plus intime du code non-erlang
    avec la machine virtuelle Erlang : les \textit{Linked-in Drivers}
  \end{alertblock}

\end{frame}
