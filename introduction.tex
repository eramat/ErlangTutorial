\section{Introduction}

\begin{frame}
  \frametitle{Introduction}

  \begin{block}{Histoire}
    \begin{itemize}
      \item créé en 1986 par Ericsson (acteur de l'industrie des
        télécommunications)
      \item libre depuis 1998
      \item le livre de référence ``Programming Erlang: Software for a
        concurrent world'' de Joe Armstrong en 2007
    \end{itemize}
  \end{block}

  \begin{exampleblock}{Philosophie}
    \begin{itemize}
      \item langage simple et épuré
      \item gestion de la concurrence très légère
      \item pas d'état partagé entre les process (pas de ``lock'', pas de mutex,
        \ldots)
      \item passage de messages asynchrones
      \item les processus doivent être en mesure de s'exécuter tout le temps
    \end{itemize}
  \end{exampleblock}

\end{frame}

\begin{frame}
  \frametitle{Introduction}

  \begin{block}{Caractéristiques du langage}
    \begin{itemize}
      \item fonctionnel
      \item typage fort et dynamique
      \item pas d'état partagé ! pas de variable mutable
      \item basé sur le ``pattern matching''
    \end{itemize}
  \end{block}

\end{frame}

\begin{frame}[fragile]
  \frametitle{Introduction}

  \begin{block}{Le shell}
    \begin{itemize}
      \item l'outil central : la console erlang
      \item le shell est un interpréteur de commandes / d'instructions
    \end{itemize}
  \end{block}

  \begin{exampleblock}{Le shell}
    \begin{lstlisting}[language=bash]
$ erl
Erlang R15B01 (erts-5.9.1) [source] [smp:2:2] [async-threads:0]
                           [kernel-poll:false]

Eshell V5.9.1  (abort with ^G)
1>
    \end{lstlisting}
  \end{exampleblock}

  \begin{alertblock}{}
    Pour quitter le shell, <ctrl> + g puis q
  \end{alertblock}

\end{frame}
