\section{Conditionnel}

\begin{frame}
  \frametitle{Conditionnel}
  \framesubtitle{Introduction}

  \begin{block}{Guard/if/case}
    Il existe plusieurs formes de conditions :
    \begin{itemize}
    \item les ``guards'' (gardes) qui permettent d'augmenter l'expressivité
      des ``patterns''
    \item les ``if'' qui est une sorte de systèmes de ``guards'' en cascade
    \item les ``case'': très similaire au ``case'' des langages classiques
    \end{itemize}
  \end{block}

\end{frame}

\begin{frame}[fragile]
  \frametitle{Conditionnel}
  \framesubtitle{Guard}

  \begin{block}{Définition}
    \begin{itemize}
    \item le ``pattern'' est utilisé si la ``guard'' est validée
    \item la ``guard'' est une fonction booléenne dépendante des arguments
      du pattern
    \item l'instruction \textit{when} permet d'exprimer la condition
    \end{itemize}
  \end{block}

  \begin{exampleblock}{Exemple}
    \begin{lstlisting}[language=erlang]
max(X, Y) when X > Y -> X;
max(X, Y) -> Y.
    \end{lstlisting}
  \end{exampleblock}

  \begin{alertblock}{Explications}
    \begin{itemize}
    \item le premier ``pattern'' est utilisé si $X > Y$
    \item sinon c'est le deuxième qui sert à l'unification
    \item les ``patterns'' sont pris dans l'ordre donc attention à l'ordre et
      au recouvrement des conditions
    \end{itemize}
  \end{alertblock}

\end{frame}

\begin{frame}[fragile]
  \frametitle{Conditionnel}
  \framesubtitle{If}

  \begin{block}{Définition}
    \begin{itemize}
    \item le ``If'' est une séquence de conditions
    \item dès que l'une est vérifiée, on applique le membre droit
    \item la séquence se termine souvent par \textit{true}
    \end{itemize}
  \end{block}

  \begin{exampleblock}{Exemple}
    \begin{lstlisting}[language=erlang]
if
    X > Y -> "X est plus grand que Y";
    true -> "X est plus petit ou égal à Y"
end.
    \end{lstlisting}
  \end{exampleblock}

  \begin{alertblock}{Explications}
    \begin{itemize}
    \item si $X > Y$ alors la chaîne de caratères "X est plus grand que Y"
      est produite
    \item si la première condition n'est pas vérifiée, on applique
      obligatoirement la seconde
    \end{itemize}
  \end{alertblock}

\end{frame}

\begin{frame}[fragile]
  \frametitle{Conditionnel}
  \framesubtitle{Case}

  \begin{block}{Définition}
    \begin{itemize}
    \item en fonction de la valeur d'une expression, on active l'une ou l'autre
      des expressions
    \item la valeur est comparée à un ``pattern''
    \item les patterns peuvent être conditionnés par des ``guards''
    \end{itemize}
  \end{block}

  \begin{exampleblock}{Syntaxe}
    \begin{lstlisting}[language=erlang]
case Expression of
    Pattern1 [when Guard1] -> Expr_seq1;
    Pattern2 [when Guard2] -> Expr_seq2;
    ...
end
    \end{lstlisting}
  \end{exampleblock}

\end{frame}

\begin{frame}[fragile]
  \frametitle{Conditionnel}
  \framesubtitle{Case}

  \begin{exampleblock}{Exemple}
    \begin{lstlisting}[language=erlang]
filter(P, [H|T]) ->
   case P(H) of
      true -> [H|filter(P, T)];
      false -> filter(P, T)
   end;
filter(P, []) -> [].
    \end{lstlisting}
  \end{exampleblock}

  \begin{alertblock}{Explications}
    \begin{itemize}
    \item la fonction \textit{filter} construit une liste à partir des éléments
      d'une autre liste qui vérifient une condition
    \item en fonction de P(H), P étant une fonction, on ajoute l'élément de la
      liste (qui est en tête) ou non
    \end{itemize}
  \end{alertblock}

\end{frame}

\begin{frame}[fragile]
  \frametitle{Conditionnel}
  \framesubtitle{Case}

  \begin{exampleblock}{Exemple d'utilisation}
    \begin{lstlisting}[language=erlang]
1> P = fun(X) -> X rem 2 == 0 end.
#Fun<erl_eval.6.82930912>
2> filter:filter(P, [1,2,3,4,5,6]).
[2,4,6]
    \end{lstlisting}
  \end{exampleblock}

  \begin{alertblock}{Explications}
    \begin{itemize}
    \item on définit une fonction qui vérifie que X est pair
    \item l'opérateur arithmétique \textit{rem} permet d'obtenir le reste
      de la division entière
    \item la fonction \textit{filter} est placée dans un module \textit{filter}
      d'où l'appel filter:filter
    \end{itemize}
  \end{alertblock}

\end{frame}
