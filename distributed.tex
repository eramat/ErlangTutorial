\section{Programmation distribuée}

\begin{frame}[fragile]
  \frametitle{Programmation distribuée}
  \framesubtitle{Introduction}

  \begin{block}{Pourquoi ?}
    \begin{itemize}
    \item la vitesse : découper une application en plusieurs parties exécutées
      sur des machines ou processeurs ou coeurs différents (on parlera de noeud)
    \item la tolérance aux fautes : augmenter la résistance aux pannes en
      faisant coopérer plusieurs noeuds qui peuvent potentiellement tomber en
      panne (possibilité de redondance)
    \item accessibilité aux ressources : certains services ne sont disponibles
      que sur certains matériels spécifiques (une base de données, par exemple)
    \item extensibilité : pouvoir ajouter des noeuds supplémentaires pour
      augmenter la capacité du système
    \item certaines applications sont intrinsèquement distribuées (système
      de chat irc, par exemple)
    \end{itemize}
  \end{block}

\end{frame}

\begin{frame}[fragile]
  \frametitle{Programmation distribuée}
  \framesubtitle{Introduction}

  \begin{block}{Définition}
    \begin{itemize}
    \item la programmation distribuée en Erlang consiste à déployer une
      application sur plusieurs machines
    \item les processus sont alors distants
    \item Erlang offre un mécanisme simple de sécurité basé sur les cookies
    \end{itemize}
  \end{block}

  \begin{alertblock}{Les fonctions BIF}
    \begin{itemize}
    \item \textit{spawn(Node, Module, Function, Args)} : création d'un
      processus sur un noeud distant
    \item \textit{monitor\_node(Node, Flag)} : vérification du bon
      fonctionnement du noeud
    \item \textit{node()} et \textit{nodes()} : nom du noeud courant et des
      noeuds communs
    \item \textit{node(Pid)} : nom du noeud où le processus identifié par Pid
      est exécuté
    \item \textit{disconnect\_node(Nodename)} : suppression de la connexion
      du noeud courant vers le noeud \textit{Nodename}
    \end{itemize}
  \end{alertblock}

\end{frame}

\begin{frame}[fragile]
  \frametitle{Programmation distribuée}
  \framesubtitle{Sécurité}

  \begin{block}{Authentification}
    \begin{itemize}
    \item lors de la communication entre 2 noeuds, un atome (le cookie) est
      échangé afin de vérifier l'authentification de l'expéditeur
    \item chaque noeud possède son propre atome (secret)
    \item par défaut, l'atome vaut \textit{nocookie} ; la fonction
      \textit{erlang:get\_cookie()} permet de le connaître
    \item \textit{erlang:set\_cookie(Node,Cookie)} définit le cookie du noeud
      courant (Node = node()) ou tente de définir le cookie d'un noeud distant
    \end{itemize}
  \end{block}

\end{frame}

\begin{frame}[fragile]
  \frametitle{Programmation distribuée}

  \begin{block}{Nommage des machines}
    \begin{itemize}
    \item afin de communiquer entre machines, il est nécessaire de nommer les
      machines
    \item on parlera de noeud (\textit{node})
    \item ce nommage est fait lors du lancement de la machine virtuelle
      Erlang
    \item deux possibilités :
      \begin{itemize}
      \item -sname xxxx : la machine appartient à un seul domaine ; on peut
        désigner une machine par un simple nom (sous Linux, on se base sur
        les fichiers hostname et hosts)
      \item -name xxxx : les machines ont besoin d'un adressage complet
        nécessitant un DNS
      \end{itemize}
    \end{itemize}
  \end{block}

  \begin{exampleblock}{Exemple}
    \begin{lstlisting}[language=bash]
eric@UltraMars:~> erl -sname node1
Erlang R15B01 (erts-5.9.1) [source] [smp:2:2] [async-threads:0] [kernel-poll:false]

Eshell V5.9.1  (abort with ^G)
(node1@UltraMars)1>
    \end{lstlisting}
  \end{exampleblock}

\end{frame}

\begin{frame}[fragile]
  \frametitle{Programmation distribuée}

  \begin{block}{Envoi de messages}
    \begin{itemize}
    \item un message est maintenant envoyé à un processus appartenant à un
      autre noeud
    \item l'un des processus doit être enregistré afin d'amorcer la
      communication
      \begin{lstlisting}[language=erlang]
{Named_processus, Named_Node} ! Message
      \end{lstlisting}
    \item l'envoi de messages peut aussi se faire de manière classique
    \item on peut connaître le nom d'un noeud via la fonction BIF \textit{node}
    \end{itemize}
  \end{block}

  \begin{exampleblock}{Exemple}
    \begin{lstlisting}[language=erlang]
{my_processus, node1@UltraMars} ! Message
    \end{lstlisting}
  \end{exampleblock}

  \begin{alertblock}{Lancement à distance}
    On peut lancer un processus sur un autre noeud :
    \begin{lstlisting}[language=erlang]
spawn(Node, Module, Function, Parameters).
    \end{lstlisting}
  \end{alertblock}

\end{frame}

\begin{frame}[fragile]
  \frametitle{Programmation distribuée}

  \begin{block}{Monitoring}
    Le test de fonctionnement se fait en 3 temps :
    \begin{itemize}
    \item déclaration du monitoring
    \item attente de l'état
    \item réception de la réponse : si on reçoit le tuple {nodedown, Node}
      alors le noeud n'est pas actif
    \end{itemize}
  \end{block}

  \begin{exampleblock}{Exemple}
    \begin{lstlisting}[language=erlang]
monitor_node(Node, true),
receive
  {nodedown, Node} ->
  ...
end,
    \end{lstlisting}
  \end{exampleblock}

  \begin{block}{Fin du monitoring}
    On peut désactiver le monitoring en faisant appel à \textit{monitor\_node}
    avec l'atome false en deuxième paramètre.
  \end{block}

\end{frame}

\begin{frame}[fragile]
  \frametitle{Programmation distribuée}

  \begin{block}{Connexion / deconnexion}
    \begin{itemize}
    \item un noeud N1 est connecté à un noeud N2, dès qu'une fonction
      invoquée par N1 mentionne N2
    \item au démarrage, la fonction \textit{nodes()} retourne la liste vide
    \item la fonction \textit{disconnect\_node} peut être utilisée pour
      supprimer le lien de connexion avec un noeud
    \end{itemize}
  \end{block}

\end{frame}

