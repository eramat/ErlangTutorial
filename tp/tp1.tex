\documentclass[11pt,a4paper,oneside]{article}
\usepackage{listings}
%\usepackage{ucs}
\usepackage[latin1]{inputenc}
\usepackage[T1]{fontenc}
\usepackage{txfonts}
\usepackage{lmodern}
\usepackage[pdftex]{thumbpdf}
\usepackage[citecolor={purple},linkcolor={blue},urlcolor={blue},
   a4paper,colorlinks,breaklinks]{hyperref}
\usepackage{txfonts}
\usepackage{hyperref}
\usepackage{xcolor}
\usepackage{graphicx}

\graphicspath{{../figures/}}

\renewcommand\familydefault{\sfdefault}

\usepackage{vmargin}
\setmarginsrb{3.3cm}{2cm}{1cm}{1cm}{1cm}{1cm}{0.5cm}{0.5cm}

%% \usepackage{fullpage}
\usepackage{color}
\usepackage{url}
\usepackage[french]{babel}
\usepackage{listings}

\author{Eric Ramat\\\url{ramat@lisic.univ-littoral.fr}}
\title{\textbf{\textbf{TP Introduction � Erlang}}\\
\emph{Programmation s�quentielle}}
\newcommand{\orangeline}{\rule{\linewidth}{1mm}}

\lstset{language=C++,extendedchars=true,inputencoding=latin1,
    basicstyle=\ttfamily\small, commentstyle=\ttfamily\color{red},
      showstringspaces=false,basicstyle=\ttfamily\small}


\newcommand{\background}{
\setlength{\unitlength}{1in}
\begin{picture}(0,0)
 \put(-1.4,-7.45){\includegraphics[height=28.7cm]{background1}}
\end{picture}}

\begin{document}
\maketitle
\background

\begin{flushright}
  Dur�e : 5 heures\end{flushright}

\noindent\orangeline

L'objectif de ce TP est de comprendre la programation en Erlang en mode
s�quentielle. Nous aborderons les types de base (num�riques, liste, tuple et
record), les fonctions et les modules. Le ``pattern matching'' sera au coeur
des exercices.

\section{Travail}

\subsection{Etape 1 : }

La g�n�ration de code, c'est bien mais maintenant tentons quelque modifications
de l'application.\\

\textbf{Exercice 1.} Ecrire une fonction
\textit{inc\_elemnet\_at\_integer\_list}
qui ajoute 1 � l'ensemble des �l�ments
d'une liste d'entiers. On v�rifiera que l'�l�ment de la liste est bien un
entier. Si ce n'est pas le cas alors il sera supprim� de la liste. La fonction
sera plac�e dans un module \textit{util}.

Les tests unitaires suivants doivent passer :
\begin{lstlisting}[language=erlang]
-module(test).
-export([]).
-import(util, [inc_element_at_integer_list/1]).

-include_lib("eunit/include/eunit.hrl").

inc_1_test() ->
?assertEqual(util:inc_element_at_integer_list([]), []).

inc_2_test() ->
?assertEqual(util:inc_element_at_integer_list([1,2]), [2,3]).

inc_3_test() ->
?assertEqual(util:inc_element_at_integer_list([1,a,[x,y,"z"]]), [2]).

inc_4_test() ->
?assertEqual(util:inc_element_at_integer_list({1,2}), {error, no_list}).
\end{lstlisting}

L'ex�cution des tests est r�alis� comme suit :
\begin{lstlisting}[language=bash]
1> c(test).
{ok,test}
2> test:test().
  All 3 tests passed.
ok
\end{lstlisting}

\textbf{Attention !!! La compilation doit �tre sans aucun warning !}\\

\textbf{Exercice 2.} Ecrire une fonction \textit{insert\_object\_in\_list} qui
ins�re un objet d�finit � l'aide des \textit{records} dans une liste selon
une politique donn�e.

\textbf{Exercice 3.} Proposez les fonctions suivantes applicables sur un arbre
binaire repr�sent� par des tuples r�cursifs.


\end{document}
