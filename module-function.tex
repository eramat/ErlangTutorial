\section{Module et fonction}

\begin{frame}[fragile]
  \frametitle{Module}

  \begin{block}{Défintion}
    \begin{itemize}
    \item un module est l'unité de base du code
    \item les fonctions doivent être définies dans un module
    \item un module est un fichier .erl
    \item un module compilé donne naissance à un fichier .beam
    \end{itemize}
  \end{block}

  \begin{exampleblock}{Exemple}
    \begin{lstlisting}[language=erlang]
-module(geometry).
-export([area/1]).
area({rectangle, Width, Height}) -> Width * Height;
area({circle, Radius}) -> 3.14159 * Radius * Radius.
    \end{lstlisting}
  \end{exampleblock}

  \begin{alertblock}{Explications}
    \begin{itemize}
    \item le module se nomme \textit{geometry}
    \item la fonction \textit{area} est utilisable à l'extérieur du module
    \end{itemize}
  \end{alertblock}

\end{frame}

\begin{frame}[fragile]
  \frametitle{Module}

  \begin{block}{Utilisation d'un module}
    \begin{itemize}
    \item après une phase de compilation :
      \begin{lstlisting}[language=erlang]
1> c(geometry).
{ok,geometry}
      \end{lstlisting}
    \item on peut ``invoquer'' les fonctions
      \begin{lstlisting}[language=erlang]
2> geometry:area({rectangle, 10, 5}).
50
3> geometry:area({circle, 1.4}).
6.15752
      \end{lstlisting}
    \end{itemize}
  \end{block}

\end{frame}

\begin{frame}[fragile]
  \frametitle{Fonction}

  \begin{block}{Définition}
    \begin{itemize}
    \item deux types de fonction : nommée et anonyme
    \item une fonction est définie par son nom et son arité (nombre de
      paramètres)
    \item deux fonctions de même nom et d'arité différent sont des fonctions
      différentes
    \item par convention, les fonctions de même nom et d'arité différent sont
      des fonctions auxiliaires
    \end{itemize}
  \end{block}

  \begin{exampleblock}{Exemple}
    \begin{lstlisting}[language=erlang]
sum(L) -> sum(L, 0).
sum([], N) -> N;
sum([H|T], N) -> sum(T, H+N).
    \end{lstlisting}
  \end{exampleblock}

\end{frame}

\begin{frame}[fragile]
  \frametitle{Fonction}
  \framesubtitle{Fonction anonyme}

  \begin{block}{Définition}
    \begin{itemize}
    \item une fonction anonyme est une fonction sans nom
    \item en général, utilisée au sein d'une expression
    \item on manipule la fonction via des variables (une sorte de pointeur
      de fonction)
    \item l'invocation de la fonction est possible via les variables
    \end{itemize}
  \end{block}

  \begin{exampleblock}{Exemple}
    \begin{lstlisting}[language=erlang]
1> Z = fun(X) -> 2*X end.
#Fun<erl_eval.6.82930912>
2> Z(2).
4
   \end{lstlisting}
  \end{exampleblock}

\end{frame}

\begin{frame}[fragile]
  \frametitle{Fonction}
  \framesubtitle{BIF}

  \begin{block}{Définition}
    \begin{itemize}
    \item Erlang propose un ensemble de fonctions prédéfinies : les BIF
      (\textit{Built-In Functions})
    \end{itemize}
  \end{block}

\end{frame}
