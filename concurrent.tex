\section{Programmation concurrente}

\begin{frame}
  \frametitle{Programmation concurrente}
  \framesubtitle{Introduction}

  \begin{block}{Définition}
    \begin{itemize}
    \item la progammation concurrente est une organisation du code en
      processus parallèles communicants
    \item en Erlang, il est possible de définir énormément de processus
    \item les processus Erlang sont des sortes de \textit{threads} allégés et
      sans partage d'état
    \item les processus communiquent par envoi de messages asynchrones
    \end{itemize}
  \end{block}

  \begin{alertblock}{C'est facile en Erlang !}
    Seulement trois primitives :
    \begin{itemize}
    \item \textit{spawn} pour la création des processus
    \item \textit{send} pour l'envoi de message
    \item \textit{receive} pour la réception des messages
    \end{itemize}
    L'absence de mémoire partagée simplifie !
  \end{alertblock}

\end{frame}

\begin{frame}[fragile]
  \frametitle{Programmation concurrente}
  \framesubtitle{Création des processus}

  \begin{block}{Définition}
    \begin{itemize}
    \item les processus sont créés par la fonction BIF \textit{spawn}
      \begin{lstlisting}[language=erlang]
spawn(Module, Fonction, Parametres)
      \end{lstlisting}
    \item les paramètres sont représentés par une liste
    \item la fonction \textit{spawn} retourne le \textit{pid} (\textit{
      Process Identifier}) du processus
    \end{itemize}
  \end{block}

  \begin{exampleblock}{Exemples}
    \begin{itemize}
    \item lancement d'un serveur défini dans le module \textit{my\_server} via
      la fonction \textit{start} avec un seul paramètre entier 8081 :
      \begin{lstlisting}[language=erlang]
Pid1 = spawn(my_server, start, [8081]).
      \end{lstlisting}
    \item lancement d'une fonction anonyme dans un processus :
      \begin{lstlisting}[language=erlang]
Pid2 = spawn(fun() -> 2 * 2 end).
      \end{lstlisting}
    \end{itemize}
  \end{exampleblock}

\end{frame}

\begin{frame}[fragile]
  \frametitle{Programmation concurrente}
  \framesubtitle{Envoi de messages}

  \begin{block}{Définition}
    \begin{itemize}
    \item chaque processus a une file d'attente de messages
    \item cette file d'attente est de type FIFO (\textit{First In Firt Out})
    \item la syntaxe d'envoi est très simple :
      \begin{lstlisting}[language=erlang]
Pid ! Message
      \end{lstlisting}
    \item le message peut être un atome, une liste, un tuple, un entier, \ldots
    \item par convention, s'il y a des données dans le message, on utilise
      un tuple dont le premier élément est un atome représentant le type
      du message
    \item pour envoyer un message à un processus, il faut connaître son pid
    \end{itemize}
  \end{block}

  \begin{exampleblock}{Exemples}
    \begin{lstlisting}[language=erlang]
Pid ! start.
Pid ! {check, 3}.
    \end{lstlisting}
  \end{exampleblock}

\end{frame}

\begin{frame}[fragile]
  \frametitle{Programmation concurrente}
  \framesubtitle{Réception des messages}

  \begin{block}{Définition}
    \begin{itemize}
    \item la réception des messages par un processus se fait via un bloc
      \textit{receive}
    \item un bloc \textit{receive} est une sorte de \textit{case} :
      \begin{lstlisting}[language=erlang]
receive
  pattern1 ->
    actions1;
  pattern2 ->
    actions2;
  ...
  patternN
    actionsN
end.
      \end{lstlisting}
    \item à la réception d'un message, le message est unifié avec le premier
      \textit{pattern} valide et l'action est réalisée
    \end{itemize}
  \end{block}

\end{frame}

\begin{frame}[fragile]
  \frametitle{Programmation concurrente}
  \framesubtitle{Réception des messages}

  \begin{block}{Définition}
    \begin{itemize}
    \item en général, le bloc \textit{receive} est défini dans une fonction
      récursivement infinie
    \item le \textit{receive} est bloquant
    \end{itemize}
  \end{block}

  \begin{exampleblock}{Exemple}
      \begin{lstlisting}[language=erlang]
loop() ->
  receive
    start ->
      % do something
      loop();
    {check, X} ->
      % do something
      loop();
    finish ->
      ok
  end.
      \end{lstlisting}
  \end{exampleblock}

\end{frame}

\begin{frame}[fragile]
  \frametitle{Programmation concurrente}
  \framesubtitle{Messages}

  \begin{block}{Asynchrone}
    \begin{itemize}
    \item par défaut, les messages sont asynchrones :
      \begin{itemize}
      \item le message est envoyé par un processus vers un autre processus
      \item il est traité dans le recepteur dès qu'il a le temps
      \item aucune réponse n'est retournée par l'opérateur !
    \end{itemize}
    \item par défaut, le récepteur ne connait pas l'expéditeur
    \end{itemize}
  \end{block}

\end{frame}

\begin{frame}[fragile]
  \frametitle{Programmation concurrente}
  \framesubtitle{Messages}

  \begin{exampleblock}{Comment implémenter des messages synchrones ?}
    \begin{itemize}
    \item en plaçant dans le message le pid de l'expéditeur via la fonction
      \textit{self} :
      \begin{lstlisting}[language=erlang]
Pid ! { self(), check, 3 },
receive
  Response ->
    Response
  end.
      \end{lstlisting}
    \item du côté du récepteur :
      \begin{lstlisting}[language=erlang]
loop() ->
  receive
    start ->
      % do something
      loop();
    {Pid, check, X} ->
      % do something
      Pid ! ok,
      loop();
    finish ->
      ok
  end.
      \end{lstlisting}
    \end{itemize}
  \end{exampleblock}

\end{frame}

\begin{frame}[fragile]
  \frametitle{Programmation concurrente}
  \framesubtitle{Processus nommé}

  \begin{block}{Définition}
    \begin{itemize}
    \item au lieu d'ajouter le pid dans le message, il est possible
      d'enregistrer un processus et d'utiliser le nommage lors d'un envoi
      de message
    \item la fonction \textit{register} est utilisée :
      \begin{lstlisting}[language=erlang]
register(some_atom, Pid)
      \end{lstlisting}
    \end{itemize}
  \end{block}

  \begin{exampleblock}{Exemple}
    \begin{lstlisting}[language=erlang]
register(my_process, spawn(my_module, my_function, [])).
...
my_process ! message
    \end{lstlisting}
  \end{exampleblock}

\end{frame}
