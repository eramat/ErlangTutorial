\section{Programmation concurrente}

\begin{frame}
  \frametitle{Programmation concurrente}
  \framesubtitle{Introduction}

  \begin{block}{Définition}
    \begin{itemize}
    \item la progammation concurrente est une organisation du code en
      processus parallèles communicants
    \item en Erlang, il est possible de définir énormément de processus
    \item les processus Erlang sont des sortes de \textit{threads} allégés et
      sans partage d'état
    \item les processus communiquent par envoi de messages asynchrones
    \end{itemize}
  \end{block}

  \begin{alertblock}{C'est facile en Erlang !}
    Seulement trois primitives :
    \begin{itemize}
    \item \textit{spawn} pour la création des processus
    \item \textit{send} pour l'envoi de message
    \item \textit{receive} pour la réception des messages
    \end{itemize}
    L'absence de mémoire partagée simplifie !
  \end{alertblock}

\end{frame}

\begin{frame}
  \frametitle{Programmation concurrente}
  \framesubtitle{Création des processus}

  \begin{block}{Définition}
    \begin{itemize}
    \item
    \end{itemize}
  \end{block}

  \begin{exampleblock}{Exemples}
  \end{exampleblock}

\end{frame}

\begin{frame}
  \frametitle{Programmation concurrente}
  \framesubtitle{Envoi de messages}

  \begin{block}{Définition}
    \begin{itemize}
    \item
    \end{itemize}
  \end{block}

  \begin{exampleblock}{Exemples}
  \end{exampleblock}

\end{frame}

\begin{frame}
  \frametitle{Programmation concurrente}
  \framesubtitle{Réception des messages}

  \begin{block}{Définition}
    \begin{itemize}
    \item
    \end{itemize}
  \end{block}

  \begin{exampleblock}{Exemples}
  \end{exampleblock}

\end{frame}
