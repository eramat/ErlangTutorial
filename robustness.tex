\section{La robustesse}

\begin{frame}
  \frametitle{La robustesse}
  \framesubtitle{Introduction}

  \begin{block}{Définition}
    \begin{itemize}
    \item tout peut arriver dans un programme :
      \begin{itemize}
      \item échec de calcul,
      \item paramètre invalide,
      \item processus mort,
      \item \ldots
      \end{itemize}
    \item il faut pouvoir réagir et traiter ces problèmes
    \item plusieurs mécanismes :
      \begin{itemize}
      \item les timeouts
      \item la capture des erreurs
      \end{itemize}
    \end{itemize}
  \end{block}

\end{frame}

\begin{frame}[fragile]
  \frametitle{La robustesse}
  \framesubtitle{Timeout}

  \begin{block}{Définition}
    \begin{itemize}
    \item lorsque l'on rentre dans une phase d'attente de message
      (\textit{receive}), on est bloqué
    \item on peut introduire des timeouts pour stopper l'attente (exprimée en
      ms)
    \end{itemize}
  \end{block}

  \begin{exampleblock}{Exemple}
    \begin{lstlisting}[language=erlang]
loop() ->
  receive
    Message ->
      io:format("Message received~n", []),
      loop()
    after 5000 ->
      io:format("Timed out~n", [])
  end.
    \end{lstlisting}
  \end{exampleblock}

  \begin{alertblock}{Bonne pratique}
    Le délai peut être une fonction.
  \end{alertblock}

\end{frame}

\begin{frame}[fragile]
  \frametitle{La robustesse}
  \framesubtitle{Capture des erreurs}

  \begin{block}{Définition}
    \begin{itemize}
    \item par défaut, un processus erlang se termine par un
      \textit{exit(normal)}
    \item si une erreur survient, la fonction \textit{exit} est invoquée avec
      un atome différent de \textit{normal}
    \item deux processus peuvent être liés et donc un processus peut savoir si
      un processus a échoué
    \item pour prendre en compte les signaux exit dès leur création, il faut
      déclarer une priorité spéciale : \texttt{process\_flag(trap\_exit, true)}
    \end{itemize}
  \end{block}

\end{frame}

\begin{frame}[fragile]
  \frametitle{La robustesse}
  \framesubtitle{Capture des erreurs}

  \begin{exampleblock}{Exemple}
    \begin{itemize}
    \item un porcessus se lie à un autre processus et possède une clause
      conduisant à l'arrêt du processus avec un atome \textit{error}
    \item dès emission d'un signal \textit{exit}, les processus liés sont
      avertis
      \begin{lstlisting}[language=erlang]
start() ->
  link(another_processus_Pid).

process([]) ->
  exit(error).
      \end{lstlisting}
    \item en cas d'erreur du premier processus, un signal est envoyé et capturé
      par un \textit{receive}
      \begin{lstlisting}[language=erlang]
loop() ->
  receive
    {message, Message} ->
      io:format("Message received~n", []),
      loop();
    {'EXIT', From, Reason} ->
      io:format("Exit: ~p~n", [{'EXIT', From, Reason}])
    end.
      \end{lstlisting}
    \end{itemize}
  \end{exampleblock}

\end{frame}
